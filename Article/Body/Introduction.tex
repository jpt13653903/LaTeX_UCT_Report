\section{Introduction}

If you are new to \LaTeX{}, I would suggest reading~\cite{Oetiker_2015}.  If you want to use Microsoft Word (or one of its many clones), you can download the official IEEE conference template from~\cite{Word_Template}.  The TA and tutors can provide \LaTeX{} support.  Use Word at your own risk.

The introduction is where you set the scene.  Here you reference other, related work, as well as a summary relating to how you improve upon said work~\cite{BibExample}.  In the sense of the practical reports, the introduction will summarise the experiment the practical is all about.

As a general rule of thumb, keep the introduction to the first column and don't put any \mbox{sub-sections} into it.

Remember that, for bibliography citations to work, you have to include running Bib\TeX{} in the compile chain.  My TeXstudio~\cite{TeXstudio} compile chain for ``Build \& View'' is\linebreak
\vspace{-6mm}
\begin{verbatim}
txs:///bibtex | txs:///pdflatex |
txs:///bibtex | txs:///pdflatex |
txs:///view-pdf-internal
\end{verbatim}
% \linebreak stretches the line that is broken to the end of the column.  Use "\\" or "\newline" if you don't want it to do this.

\subsection{Drafting Markup}

When the template is in draft mode, you can use various helper macros, as illustrated below:

\old{This is old text that should be removed.}  \note{This is a note about something to remember, or comments from the proof-reader.}  \todo{This is something that still needs doing.}  When compiled with \verb|\Draftfalse|, the content of these macros are removed from the output, \rephrase{except something that needs to be rephrased.}

\note{You can also use cards, as follows:}

\todocard{
  This is a todo card.
  
  It is a minipage environment, so you can have all sorts of stuff in it.  It can be many paragraphs long, but don't make it too long, because \LaTeX\ will force the whole card onto a single page.

  \notecard{This is a nested note card.  You can nest cards of arbitrary types as deep as you like.}
}

