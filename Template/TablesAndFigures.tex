\newlength{\TempFigureLength}
%-------------------------------------------------------------------------------

\newenvironment{Float}{
  \ifFloat
    \figure[!t]
  \else
    \figure[!htbp]
  \fi
  \centering
}{
  \endfigure%
  \ifFloat\else
    \FloatBarrier
  \fi
}

\newcommand{\PrepareFigureCaption}[2]{
  \def\FigureCaption{#1}
  \refstepcounter{figure}
  \addcontentsline{lof}{section}{\makebox[20mm][l]{Fig.~\thefigure{}}~\FigureCaption}
  \label{fig:#2}
}
\newcommand{\PlaceFigureCaption}{%
  {%
    \ifTwoColumn%
      \footnotesize%
    \else%
      \small%
    \fi%
    \settowidth{\TempFigureLength}{Fig.~\thefigure{}.~~{\FigureCaption}}%
    \ifdim \TempFigureLength > 0.95\columnwidth%
      \parbox{0.95\columnwidth}{Fig.~\thefigure{}.~~{\FigureCaption}}%
    \else%
      Fig.~\thefigure{}.~~{\FigureCaption}%
    \fi%
  }%
}

\newenvironment{FigureEnvironment}[2]{
  \begin{Float}%
  \PrepareFigureCaption{#1}{#2}
}{
  \PlaceFigureCaption%
  \end{Float}%
}
%-------------------------------------------------------------------------------

\def\FigureSize{1.2}
\ifTwoColumn
  \def\FigureSize{0.8}
\fi

\newcommand{\Figure}[3][scale=\FigureSize]{%
  \begin{FigureEnvironment}{#2}{#3}%
    \includegraphics[#1]{../Figures/#3}\\[1em]%
  \end{FigureEnvironment}%
}
%-------------------------------------------------------------------------------

\newenvironment{TableEnvironment}[2]{
  \ifFloat
    \begin{table}[!t]%
  \else
    \begin{table}[!htbp]%
  \fi
  \begin{center}%
  \refstepcounter{table}
  \addcontentsline{lot}{section}{TABLE \thetable{}~~~#1}
  \label{tab:#2}%
  TABLE \thetable{}\\[1ex]
  \settowidth{\TempFigureLength}{\textsc{#1}}
  \ifdim \TempFigureLength > 0.95\columnwidth
    \parbox{0.95\columnwidth}{\textsc{#1}}%
  \else
    \textsc{#1}%
  \fi\\[1em]
  \renewcommand{\arraystretch}{1.2}
}{
  \end{center}%
  \end{table}%
  \ifFloat\else
    \FloatBarrier
  \fi
}
%-------------------------------------------------------------------------------

\newcommand{\Table}[4]{%
  \begin{TableEnvironment}{#1}{#4}%
    \begin{tabular}{#2}
      #3
    \end{tabular}
  \end{TableEnvironment}%
}
%-------------------------------------------------------------------------------
