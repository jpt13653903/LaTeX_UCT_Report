\Chapter{Methodology}

In this section you should describe the method followed in your research project, which for an engineering case is typically the development and experimentation of the system that is developed to solve a problem together with how testing of the system was carried out to show that it indeed solved the problem or at least to what extent it solved the problem (note that, yes, unfortunately it is sometimes not possible to provide an effective solution -- it may be too expensive or take to long to solve properly -- but even then it can still be considered a contribution of knowledge and still have some value as a research investigation).

\textbf{Methodology vs Design:} A vitally important terminology consideration is to realize that  methodology is not the same as design. This is a common confusion amongst students who have not done a research project or written a thesis before. This short explanation should suffice to ensure you understand the difference:

\begin{itemize}
	\item Methodology: this describes how you are going to go about doing the research project. It describes the main high-level steps (or phases) of the project. It does not detail design or pieces of your design. You should not need any block diagrams in the design.
	\item Design: this is what you have built (or are going to build). It includes diagrams, system block diagram, flow charts, pseudocode/algorithms, schematics, etc. Generally, in a computing-related project, it will end off with implementation details (actual code snippets). These implementation details, depending on how much there are of them, may be better placed in a separate Implementation chapter. Similarly, you might decide to have a separate Integration chapter if there are a lot of pieces that takes quite a bit of explaining and diagrams to describe how they connect and work together. 
\end{itemize}



\section{Experiment Procedure}

Furthermore, include detail relating to the experiment itself: what did you do, in what order was this done, why was this done, etc.  What are you trying to prove / disprove?  You can include hypotheses, such as presented in Hypothesis~\ref{hyp:Example} below.

\Hypothesis{Example}{All scientific papers contain hypotheses.  An hypothesis is generally not longer than a single paragraph, but the command does support multiple paragraphs if required.}
