\Chapter{Conclusion}
\label{Ch:Conclusion}

The conclusion should provide a summary of your findings.  Keep in mind that people sometimes only read the introduction and conclusion of an academic work, which are the most interesting parts.  They sometimes scan the tables and figures in-between to get a quick overview of specific things done.  If the conclusion hints at interesting findings, only then will the reader be more likely go back and read the whole thesis or paper.

You can also include work that you intend to do in future or would like to recommend others to do, specifically ideas for further improvements, or to make the solution more accessible to the general user-base.

These are some tips for what a conclusion should provide:

\begin{itemize}
  \item Reflect on what was done, this may be a recap of the main tasks carried out.

  \item Your conclusion should refer back to the original objectives / hypothesis or research questions and explain how they were achieved or why they weren't (or why the original objectives were ineffectively posed and better ones were developed).

  \item Discuss hypotheses and how they were proven or disproved.  This is optional -- only applicable if hypothesis approach was followed (and isn't redundant in terms of the above point).

  \item `Last word(s)' on the project, \eg~the most important conclusion or something sweeping (and preferably ending on a positive issue) as a final point.

  \item Future work -- usually you don't need to have a whole chapter dedicated to this.  A couple paragraphs (say a half-page for MSc and whole page for PhD) are sufficient.

  \item You might (if you are one to strive for perfection) want to make the very last sentence/paragraph something elegant and memorable so that your manuscript ends on a high point, \eg~``Based on the unexpected results of X it is clearly a need for Y that may lead, in the future, to Z'', or something along those lines.
\end{itemize}
